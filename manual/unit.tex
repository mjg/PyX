\chapter{Module unit}
\label{unit}

With the \verb|unit| module \PyX{} makes available classes and
functions for the specification and manipulation of lengths. As usual,
lengths consist of a number together with a measurement unit, e.g.,
\unit[1]{cm}, \unit[50]{points}, \unit[0.42]{inch}.  In addition,
lengths in \PyX{} are composed of the five types ``true'', ``user'',
``visual'', ``width'', and ``\TeX'', e.g., \unit[1]{user cm},
\unit[50]{true points}, $(0.42\ \mathrm{visual} + 0.2\ 
\mathrm{width})$ inch.  As their names indicate, they serve different
purposes. True lengths are not scalable and are mainly used for return
values of \PyX{} functions.  The other length types can be rescaled by
the user and differ with respect to the type of object they are
applied to:

\begin{description}
\item[user length:] used for lengths of graphical objects like
  positions etc.
\item[visual length:] used for sizes of visual elements, like arrows,
  graph symbols, axis ticks, etc.
\item[width length:] used for line widths
\item[\TeX{} length:] used for all \TeX{} and \LaTeX{} output
\end{description}

For instance, if you only want thicker lines for a publication
version of your figure, you can just rescale the width lengths. How
this all works, is described in the following sections.

\section{Class length}

The constructor of the \verb|length| class accepts as first argument
either a number or a string:
\begin{itemize}
\item \verb|length(number)| means a user length in units of the default
unit, defined via \verb|unit.set(defaultunit=defaultunit)|.
\item For \verb|length(string)|, the \verb|string| has to consist of a
  maximum of three parts separated by one or more whitespaces:
\begin{description}
\item[quantifier:] integer/float value. Optional, defaults to \verb|1|.
\item[type:] \verb|"t"| (true), \verb|"u"| (user), \verb|"v"|
  (visual), \verb|"w"| (width),
  or \verb|"x"| (TeX).
  Optional, defaults to \verb|"u"|.
\item[unit:] \verb|"m"|, \verb|"cm"|, \verb|"mm"|, \verb|"inch"|, or
  \verb|"pt"|. Optional, defaults to the default unit.
\end{description}
\end{itemize}
The default for the first argument is chosen in such a way that
\texttt{5*length()==length(5)}.  Note that the default unit  is
initially set to \verb|"cm"|, but can be changed at any time by the
user. For instance, use
\begin{quote}
\begin{verbatim}
unit.set(defaultunit="inch")
\end{verbatim}
\end{quote}
if you want to specify per default every length in inches.
Furthermore, the scaling of the user, visual and width types can be
changed with the \verb|set| function, as well. To this end, \verb|set|
accepts the named arguments \verb|uscale|, \verb|vscale|, and
\verb|wscale|. For example, if you like to change the thickness of all
lines (with predefined linewidths) by a factor of two, just insert
\begin{quote}
\begin{verbatim}
unit.set(wscale = 2)
\end{verbatim}
\end{quote}
at the beginning of your program.

To complete the discussion of the \verb|length| class, we mention,
that as expected \PyX{} lengths can be added, subtracted, multiplied by
a numerical factor, converted to a string and compared with each other.

\section{Subclasses of length}

A number of subclasses of \verb|length| are already predefined.  They
only differ by their defaults for \verb|type| and \verb|unit|. Note
that again the default value for the quantifier is \verb|1|, such
that, for instance, \texttt{5*m(1)==m(5)}.

\medskip
\begin{center}
\begin{tabular}{lll|lll}
Subclass of \texttt{length} & Type & Unit & Subclass of \texttt{length} & Type & Unit\\
\hline
\texttt{m(x)} & user & m & \texttt{v\_m(x)} & visual & m\\
\texttt{cm(x)} & user & cm & \texttt{v\_cm(x)} & visual & cm\\
\texttt{mm(x)} & user & mm & \texttt{v\_mm(x)} & visual & mm\\
\texttt{inch(x)} & user & inch & \texttt{v\_inch(x)} & visual & inch\\
\texttt{pt(x)} & user & points & \texttt{v\_pt(x)} & visual & points\\
\texttt{t\_m(x)} & true & m & \texttt{w\_m(x)} & width & m\\
\texttt{t\_cm(x)} & true & cm & \texttt{w\_cm(x)} & width & cm\\
\texttt{t\_mm(x)} & true & mm & \texttt{w\_mm(x)} & width & mm\\
\texttt{t\_inch(x)} & true & inch & \texttt{w\_inch(x)} & width & inch\\
\texttt{t\_pt(x)} & true & points & \texttt{w\_pt(x)} & width & points\\
\texttt{u\_m(x)} & user & m & \texttt{x\_m(x)} & \TeX & m \\
\texttt{u\_cm(x)} & user & cm & \texttt{x\_cm(x)} & \TeX & cm \\
\texttt{u\_mm(x)} & user & mm & \texttt{x\_mm(x)} & \TeX & mm \\
\texttt{u\_inch(x)} & user & inch & \texttt{x\_inch(x)} & \TeX & inch \\
\texttt{u\_pt(x)} & user & points & \texttt{x\_pt(x)} & \TeX & points\\

\end{tabular}
\end{center}
\medskip
Here, \verb|x| is either a number or a string, which, as mentioned
above, defaults to \texttt{1}.

\section{Conversion functions}
If you want to know the value of a \PyX{} length in certain units, you
may use the predefined conversion functions which are given in the
following table
\begin{center}
\begin{tabular}{ll}
function & result \\
\hline
\texttt{tom(l)} & \texttt{l} in units of m\\
\texttt{tocm(l)} & \texttt{l} in units of cm\\
\texttt{tomm(l)} & \texttt{l} in units of mm\\
\texttt{toinch(l)} & \texttt{l} in units of inch\\
\texttt{topt(l)} & \texttt{l} in units of points\\
\end{tabular}
\end{center}
If \verb|l| is not yet a \verb|length| instance, it is converted first
into one, as described above. You can also specify a tuple, if you
want to convert multiple lengths at once.


%\section{Examples}


%\subsection{Example 1}



%%% Local Variables:
%%% mode: latex
%%% TeX-master: "manual.tex"
%%% End:
