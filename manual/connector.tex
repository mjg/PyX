\chapter{Module \module{connector}}
\label{connector}

This module provides classes for connecting two \class{box}-instances with
lines, arcs or curves.
All constructors of the following connector-classes take two
\class{box}-instances as first arguments. They return a
\class{normpath}-instance from the first to the second box, starting/ending at
the boxes' outline. The behaviour of the path is determined by the
boxes' center and some angle- and distance-keywords. The resulting \class{connector} will
additionally be shortened by lengths given in the \keyword{boxdists}-keyword (a
list of two lengths, default \code{[0,0]}).

\section{Class \class{line}}

The constructor of the \class{line} class accepts only boxes and the
\keyword{boxdists}-keyword.

\section{Class \class{arc}}

The constructor also takes either the \keyword{relangle}-keyword or a
combination of \keyword{relbulge} and \keyword{absbulge}. The ``bulge'' is the
meant to be a hint of the greatest distance between the connecting arc and the
straight connecting line. (Default: \code{relangle=45},
\code{relbulge=None}, \code{absbulge=None})\medskip

Note that the bulge-keywords override the angle-keyword. When both
\keyword{relbulge} and \keyword{absbulge} are given they will be added.

\section{Class \class{curve}}

The constructor takes both angle- and bulge-keywords. Here, the bulges are
used as distances between bezier-curve control points:\medskip

\keyword{absangle1} or \keyword{relangle1}\\
\keyword{absangle2} or \keyword{relangle2}, where the absolute angle overrides the
relative if both are given. (Default: \code{relangle1=45},
\code{relangle2=45}, \code{absangle1=None}, \code{absangle2=None})\medskip

\keyword{absbulge} and \keyword{relbulge}, where they will be added if both are
given.\\ (Default: \code{absbulge=None}, \code{relbulge=0.39}; these default
values produce output similar to the defaults of \class{arc}.)\medskip

Note that relative angle-keywords are counted in the following way:
\keyword{relangle1} is counted in negative direction, starting at the straight
connector line, and \keyword{relangle2} is counted in positive direction.
Therefore, the outcome with two positive relative angles will always leave the
straight connector at its left and will not cross it.

\section{Class \class{twolines}}

This class returns two connected straight lines. There is a vast variety of
combinations for angle- and length-keywords. The user has to make sure to
provide a non-ambiguous set of keywords:\medskip

\keyword{absangle1} or \keyword{relangle1} for the first angle,\\
\keyword{relangleM} for the middle angle and\\
\keyword{absangle2} or \keyword{relangle2} for the ending angle.
Again, the absolute angle overrides the relative if both are given. (Default:
all five angles are \code{None})\medskip

\keyword{length1} and \keyword{length2} for the lengths of the connecting lines.
(Default: \code{None})

